\documentclass{beamer}
\usepackage{beamerthemeshadow}
\usepackage{graphicx}
\usepackage{color}
\usepackage[utf8]{inputenc}
\usepackage{hyperref}
\usepackage[flushleft]{threeparttable}
\usepackage{multicol} %da bi napisalo formule za amdalov zakon jedna do druge

\definecolor{beamer@boja}{rgb}{0.5,0.1,0.5}
\setbeamercolor{structure}{fg=beamer@boja}

\def\d{{\fontencoding{T1}\selectfont\dj}}
\def\D{{\fontencoding{T1}\selectfont\DJ}}


\title{Paralelno programiranje}
\subtitle{-- Seminarski rad --}
\author{Aleksandar Djukić \\Marija Zrnić \\Iva Minić \\Luka Mitrović}
\institute{Matematički fakultet\\Univerzitet u Beogradu}
\date{
	\footnotesize{Beograd, 2022.}	
}

\begin{document}
	\begin{frame}
		\thispagestyle{empty}
		\titlepage
	\end{frame}
	
	\addtocounter{framenumber}{-1}
	
	\begin{frame}[fragile]\frametitle{Literatura}
		\begin{itemize}
			\item Zasnovano na:\\
			ubaciti literaturu ovde
			%Goran Nenadic, Predrag Janičić, Aleksandar Samardžić: \LaTeX{} za autore, Beograd, Kompjuter biblioteka, 2003.
			%(\url{http://poincare.matf.bg.ac.rs/~janicic//latex2e/})
		\end{itemize}
	\end{frame}
	
	\begin{frame}
		\frametitle{Pregled}
		\tableofcontents[hidesubsections] 
	\end{frame}
	
	\section{Uvod i istorija}
	
	\begin{frame}[fragile]\frametitle{Uvod i istorija}
		\begin{itemize}	
			\item Pojam paralelnog programiranja
			\item Razlika izmedju paralelne obrade i paralelnog programiranja
			\bigskip
			\item Paralelna obrada se prvi put pominje 1950-ih
			\item 1962. Burroughs proizvodi sistem D825 sa 4 procesora
			\item 1967. Amdalov zakon nastaje na \emph{Spring Joint} konferenciji
			\item 1983. \emph{Caltech} proizvodi \emph{The Cosmic Cube} sa 64 procesora
			\bigskip
			\item 1992. \emph{Standards for Message-Passing in a Distributed Memory Environment} radionica; 1994. \emph{MPI} interfejs
			\item 1997. \emph{OpenMP Architecture Review Board} objavljuje interfejs \emph{OpenMP} za \emph{Fortran}, zatim za C/C++.
		\end{itemize}
	\end{frame}
	
	
	%\subsection{Princip rada}
	\section{Princip rada paralelnog programiranja}
	\subsection{Osnove i forme}
	\begin{frame}[fragile]\frametitle{Osnove i forme paralelnog programiranja}
		\begin{itemize}	
			\item Opšte rečeno, paralelno programiranje radi tako što deli program na više delova koji se izvršavaju istovremeno.			
			\item Da bismo razumeli celu sliku, moramo prvo objasniti paralelnu obradu:
			%ovde moze da ide ona tabela sa formama samo skraceno
		\end{itemize}
	\end{frame}

	\subsection{Amdalov zakon, koraci paralelizacije}
	\begin{frame}[fragile]\frametitle{Amdalov zakon i koraci paralelizacije programa}
	\begin{itemize}	
		\item Amdalov zakon meri efikasnost izvršavanja paralelnog programa.
	\end{itemize}
	\begin{multicols}{2}
		\begin{equation}
			S_{max} = \frac{1}{(1 - p) + \frac{p}{s}}
		\end{equation}\break
		\begin{equation}
			S_{max} = N + (1 - N) * s
		\end{equation}
	\end{multicols}
	\bigskip
	\begin{itemize}	
		\item Koraci paralelizacije programa:
		\begin{itemize}	
			\item dekompozicija
			\item dodela
			\item orkestracija
			\item mapiranje
		\end{itemize}
	\end{itemize}
	\end{frame}
	% predlog - ispod koraka staviti modele i izbaciti sledeci slajd (tesno je ali moze da stane)
	\subsection{Modeli paralelnog programiranja}
	\begin{frame}[fragile]\frametitle{Modeli paralelnog programiranja}
		\begin{itemize}	
			\item Termin \emph{komunikacija} se često pominje tokom koraka paralelizacije, a on se zapravo odnosi na razmenjivanje podataka izmedju procesa koji se trenutno izvršavaju.
			\item Na osnovu ovoga nastalo je nekoliko modela:
			\begin{itemize}	
				\item Model deljene memorije (\emph{shared memory model})
				\item Model prenošenja poruke (\emph{message passing model})
				\item Particionisani globalni adresni prostor (\emph{partitioned global address space}) -- hibridni model
			\end{itemize}
		\end{itemize}
	\end{frame}
	
	\section{Upotrebe, prednosti i mane}
	\begin{frame}[fragile]\frametitle{Upotrebe, prednosti i mane}
		\begin{itemize}	
			\item Paralelno programiranje koristimo kada imamo velike količine podataka, kompleksne račune ili velike simulacije.
			\item Neke od oblasti u kojima se paralelno programiranje koristi su primenjena fizika, elektrotehnika, finansijsko i ekonomsko modeliranje, veštačka inteligencija, kvantna mehanika i druge.
			\bigskip
			\item Prednosti paralelnog programiranja: brzina, poboljšan GUI (\emph{graphic user interface}), istovremeno pokretanje različitih logika programirnja, bolje korišćenje keš memorije i CPU resursa.
			\item Mane paralelnog programiranja: promena konteksta, nepredvidljivost, otežano programiranje, \emph{data race} i \emph{deadlock}.
		\end{itemize}
	\end{frame}
	
	\section{Primeri}
	%primeri
	

	%\section{Zakljucak}
	%\begin{frame}[fragile]\frametitle{Zaključak}
		%zakljucak
	%\end{frame}
	
\end{document}
